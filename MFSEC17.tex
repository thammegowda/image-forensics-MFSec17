\documentclass[sigconf]{acmart}

\usepackage{booktabs} % For formal tables

% Copyright
%\setcopyright{none}
%\setcopyright{acmcopyright}
%\setcopyright{acmlicensed}
%\setcopyright{rightsretained}
%\setcopyright{usgov}
%\setcopyright{usgovmixed}
%\setcopyright{cagov}
%\setcopyright{cagovmixed}


%Conference
%\copyrightyear{2017} 
%\acmYear{2017} 
%\setcopyright{acmcopyright}
%\setcopyright{rightsretained}
%\acmConference{MFSec '17}{}{June 6, 2017, Bucharest, Romania}
%\acmPrice{15.00.}
%\acmISBN{ISBN 978-1-4503-5034-1/17/06} 

%% DOI Update XXXX from authors review
%\acmDOI{http://dx.doi.org/10.1145/XXXXXXX.XXXXXXX}

\copyrightyear{2017}
\acmYear{2017}
\setcopyright{acmcopyright}
\acmConference{MFSec'17}{June 06, 2017}{Bucharest, Romania}
\acmPrice{15.00}
\acmDOI{http://dx.doi.org/10.1145/3078897.3080536} 
\acmISBN{978-1-4503-5034-1/17/06}



\graphicspath{{./images/}}
\usepackage{listings}
\usepackage{url}
\usepackage{tabularx}
\usepackage{booktabs}
\usepackage{stfloats}
\usepackage[export]{adjustbox}
\usepackage{balance}

%%%% Bullets inside table %%%%
\newcommand{\tabitem}{~~\llap{\textbullet}~~}
%%%% Reduce font between image and caption, headings
\usepackage[font=small,skip=0pt]{caption}

%%%%%%%%% JSON BEGIN %%%%%
\usepackage{xcolor}
\colorlet{punct}{red!60!black}
\definecolor{background}{HTML}{EEEEEE}
\definecolor{delim}{RGB}{20,105,176}
\colorlet{numb}{magenta!60!black}

\lstdefinelanguage{json}{
	basicstyle=\normalfont\ttfamily,
	numbers=left,
	numberstyle=\scriptsize,
	stepnumber=1,
	numbersep=8pt,
	showstringspaces=false,
	breaklines=true,
	frame=lines,
	backgroundcolor=\color{background},
	literate=
	*{0}{{{\color{numb}0}}}{1}
	{1}{{{\color{numb}1}}}{1}
	{2}{{{\color{numb}2}}}{1}
	{3}{{{\color{numb}3}}}{1}
	{4}{{{\color{numb}4}}}{1}
	{5}{{{\color{numb}5}}}{1}
	{6}{{{\color{numb}6}}}{1}
	{7}{{{\color{numb}7}}}{1}
	{8}{{{\color{numb}8}}}{1}
	{9}{{{\color{numb}9}}}{1}
	{:}{{{\color{punct}{:}}}}{1}
	{,}{{{\color{punct}{,}}}}{1}
	{\{}{{{\color{delim}{\{}}}}{1}
	{\}}{{{\color{delim}{\}}}}}{1}
	{[}{{{\color{delim}{[}}}}{1}
	{]}{{{\color{delim}{]}}}}{1},
}
%%%%%%%%%%%%%%%%%%%%JSON END %%%%

\begin{document}
\title[Automatic Large Scale Image Forensics]{An Approach for Automatic and Large Scale Image Forensics} %  and Search for the Deep Web
\author{Thamme Gowda\textsuperscript{1,2}, Kyle Hundman\textsuperscript{2}, Chris A. Mattmann\textsuperscript{1,2}}
\email{thammegowda.n@usc.edu}
\affiliation{\begin{tabular}{*{2}{>{\centering}p{.45\textwidth}}}\textsuperscript{1}Computer Science Department & \textsuperscript{2}Jet Propulsion Laboratory \tabularnewline University of Southern California & California Institute of Technology \tabularnewline Los Angeles, CA 90089 USA & Pasadena, CA 91109 USA \end{tabular}}

% The default list of authors is too long for headers}
\renewcommand{\shortauthors}{T. Gowda, K. Hundman, C. Mattmann}

%\titlenote{Apache Tika is a open source project hosted on Apache Software Foundation and Tensorflow is another opensource framework for deeplearning}
%\subtitle{Memex program}
%\%subtitlenote{Credits to DARPA Memex}


% The default list of authors is too long for headers}
%\renewcommand{\shortauthors}{T. Gowda and C. Mattmann}

\begin{abstract}
This paper describes the applications of deep learning-based image recognition in the DARPA Memex program and its repository of 1.4 million weapons-related images collected from the Deep web. We develop a fast, efficient, and easily deployable framework for integrating Google's Tensorflow framework with Apache Tika for automatically performing image forensics on the Memex data. Our framework and its integration are evaluated qualitatively and quantitatively and our work suggests that automated, large-scale, and reliable image classification and forensics can be widely used and deployed in bulk analysis for answering domain-specific questions.
\end{abstract}

%
% The code below should be generated by the tool at
% http://dl.acm.org/ccs.cfm
% Please copy and paste the code instead of the example below. 
%
\begin{CCSXML}
<ccs2012>
<concept>
<concept_id>10002951.10003317</concept_id>
<concept_desc>Information systems~Information retrieval</concept_desc>
<concept_significance>500</concept_significance>
</concept>
<concept>
<concept_id>10002951.10003317.10003371.10003386.10003387</concept_id>
<concept_desc>Information systems~Image search</concept_desc>
<concept_significance>500</concept_significance>
</concept>
<concept>
<concept_id>10010405.10010462</concept_id>
<concept_desc>Applied computing~Computer forensics</concept_desc>
<concept_significance>300</concept_significance>
</concept>
</ccs2012>
\end{CCSXML}

\ccsdesc[500]{Information systems~Information retrieval}
\ccsdesc[500]{Information systems~Image search}
\ccsdesc[300]{Applied computing~Computer forensics}

% We no longer use \terms command
%\terms{Theory}

\keywords{Image Recognition, Multimedia Forensics, Information Retrieval}

\maketitle

\section{INTRODUCTION}
%% TODO: introduce Content Analysis and role in DARPA MEMEX
Over the past two years, our research team has borne witness to the ease and availability of potentially criminal goods and services on the modern Internet. In particular, our team's work on the DARPA Memex project has focused on the issues of online gun sales, as such sales can have grim consequences in that they provide a medium for buyers and sellers to circumvent traditional background checks. In turn, this proliferates the sale of dangerous semi-automatic weapons and can lead directly to loss of human life. For instance, a New York Police Department (NYPD) investigation in 2013 identified guns used in one suicide and four murders and traced their origin to transactions on the website \url{armslist.com}\cite{raja_2016}. 

Unfortunately, law enforcement agencies don't have the manpower to effectively monitor the current scale of online weapons sales. And though the ads, like the whole Internet, contain a large amount of text \cite{mphillips-EOT2012}, the proliferation of images necessitates object recognition and image analysis at scale. Our recent work in DARPA's Memex initiative has focused on expanding Apache Tika, a content detection and analysis framework \cite{mattmann2011tika}, to support such analysis.

The ability to rapidly and automatically detect and analyze gun sales transactions is a significant challenge. There are hundreds of both national and regional gun sales sites like Armslist, \url{floridaguntrader.com}, or \url{gunbroker.com}. Besides the sheer number of sites, many of the sites share common themes indicative of today's {\em Deep} web. They require a login to either buy or sell, making bulk analysis difficult for traditional web crawlers. A significant number of the sites use AJAX or Javascript for pagination, or for displaying gun images from an ad; this also makes automatic analysis a challenge. But most significantly, the actual content required to determine the answers to significant questions regarding these weapons (``is this an automatic weapon?'', ``is this a long or a short gun?'', ``are there multiple weapons being sold?'') are the {\em images} of the weapons themselves. 

We have previously worked on bulk image analysis from the Deep web as it relates to human trafficking data \cite{mattmann7tg} using the Apache Tika. However, our work there focused on image metadata forensics as an alternative to image-pixel based analyses and object detection and recognition. Though metadata forensics were promising in human trafficking, weapons required pixel-based analyses. Based on our study of over 80 websites and online forums that specialize in the exchange of weapons, object recognition and computer vision were needed to automatically discern whether or not the guns being sold are automatic or semi-automatic, whether they have been stolen (using serial-number identification), and whether the transactions are potentially illegal. Automatically being able to discern these types of object properties in bulk analyses of image data has the potential to thwart crimes and, ultimately, to save lives.

Historically, the best object recognition systems were inaccurate, but this has changed due to recent advancements in deep neural networks, larger training datasets, and improved computing resources. Tensorflow is a scalable, Python-based system and it natively supports image recognition via its \textit{Inception} model \cite{abadi2016tensorflow}. \textit{Inception} provides a neural network trained on the ImageNet corpus \cite{krizhevsky2012imagenet}, a dataset of 14,197,122 images classified using text from the WordNet taxonomy. The end result is a highly-scalable, off-the-shelf system that can accurately identify and classify objects in images into a thousand categories. This capability -- combined with Apache Tika's native support for detecting thousands of file formats and extracting their metadata and textual content -- is an attractive, automated solution that can perform bulk analysis in the weapons domain, but more generally, in any context where text and images are present and such analyses are required.

This integration of Tensorflow with Tika presented a significant challenge: Tensorflow does not provide default bindings to Java-based frameworks. Apache Tika is primarily written in Java and thus integrating with Tensorflow is not straightforward like with other JVM-compatible libraries. Our research directly addresses this and contributes several methods that make Tensorflow easier to integrate into Java-based systems like Tika, and any digital forensics system that can make a call to an application programming interface (API). In this paper, we report on our integration of Tika and Tensorflow using the Weapons domain as a motivating example. We also evaluate the integration in both its robustness in objection recognition with zero training beyond that of ImageNet. Lastly, we demonstrate that Tensorflow and Tika together form  a scalable forensics solution for bulk image analysis on the Deep web.

The remainder of this paper is organized as follows. Section 2 discusses the collection of the Weapons dataset and its properties. Section 3 presents our integration of Tika and Tensorflow via three methods: (1) command line invocation; (2) Google's RPC (gRPC) integration; and (3) Representational Entity State Transfer (REST) \cite{Fielding:2000:ASD:932295} integration. Section 4 qualitatively evaluates our integration techniques and quantitatively evaluates the efficacy of Tensorflow, ImageNet and Tika-based image forensics. Section 5 rounds out the paper.
%\begin{figure*}[!t]
%        \centering
%	\includegraphics[width=\textwidth,height=6cm]{interface-diagram}
%	\caption{This diagram demonstrates how the integration of Tika and Tensorflow facilitates in-depth search across heterogenuous content types. There are several extensions to our object recognition implementation as well, including more refined categories, optical-character recognition, and image similarity metrics.}
%	\label{fig:interface-diagram}
%\end{figure*}
\section{Data Collection and Content Analysis in Memex}
\label{sec:memex-data}
% This section discusses motivations behind DARPA Memex and provides background on program data. 
Current commercial search engines provide generalized search interfaces that allow users to search across a limited portion of the web~\cite{fbo-memex}. However, in the context of cyber security and law enforcement, commercial search engines miss essential content from the Deep and Dark web. Because it is hard to reach, this content often harbors illicit activity. The goal of Memex is to develop software that can quickly and thoroughly collect, organize, and search subsets of information relevant to individual domains of interest. 

The program's initial focus was on human trafficking and the trade and sale of illicit goods, and several enhanced web crawlers were used to discover and retrieve information from the websites related to these domains. Along with the general-purpose web crawlers, specialized crawlers were used for the retrieval of Dark web data using The Onion Router (TOR) protocol \cite{mentor2016onion} and also specialized in the retrieval of dynamic AJAX content guarded by login forms. Fetched data were then cached within the system for analysis due to the ephemeral nature of the source (web) content.

% \subsection{Memex Dataset} \label{sec:memex-dataset}
The Memex data inlcuded 7.2 million items of content in the illegal weapon sales domain, of which 1.4 million objects were images. Before processing images, we analyzed the textual documents in a separate experiment using named-entity recognition (NER) models that extracted people, locations, organizations, weapon names, and weapon types. Tika has recently added support for this task using popular Natural Language Processing toolkits like Stanford CoreNLP\cite{Finkel:2005:INI:1219840.1219885}, Apache OpenNLP\cite{ApacheOpenNLP}, and MIT Lincoln Lab's MITIE \cite{MITIE-github}. However, as we described in Section 1, our work has focused on weapons images for two primary reaons: (1) web crawlers generally extract any linked content from a site and ensuring that images contain relevant objects was an important preprocessing step; and (2) classifying image objects allows for the cataloguing of specific objects of interest. 
With regard to number two, the queries we sought to answer were related to automated identification of (semi-)automatic weapons and illegal gun transactions -- this often requires direct analysis of the image rather than associated ad text. 

The ultimate goal of integrating and characterizing diverse content scattered across the web is to provide law enforcement analysts with tools that will help them quickly identify potentially illegal activity, and images and videos often provide salient information not available in text; in many cases, illegal weapons dealers intentionally embed revealing details in rich content mediums because they are harder to identify. This thinking coincides with the rise of weapons trafficking on social media platforms such as Snapchat, YouTube, and Instagram, where communication is centered around images and video \cite{socialmedia}. The integration of Tensorflow and Tika provides a single, streamlined platform that unites the extraction of textual and rich content. This combined content can then be exposed through search and visualization interfaces that improve analysts' abilities to drill-down and explore comprehensive, diverse content contained in weapons ads (see Figure \ref{fig:interface-diagram}). 

\begin{figure*}[!t]
        \includegraphics[width=\textwidth,height=5cm]{tensorflow-tika-integration}
        \caption{Tika and Tensorflow Integration}
        \label{fig:tf-tika-integration}
\end{figure*}


\begin{figure*}
	\includegraphics[width=\textwidth,height=6cm]{interface-diagram}
	\caption{This diagram demonstrates how the integration of Tika and Tensorflow facilitates in-depth search across heterogenuous content types. There are several extensions to our object recognition implementation as well, including more refined categories, optical-character recognition, and image similarity metrics.}
	\label{fig:interface-diagram}
\end{figure*}

\section{CONTENT ANALYSIS IN MEMEX PROGRAM} \label{sec:memex}
Law enforcement agencies don't have the manpower to effectively monitor the scale of weapons ads discussed in section \ref{memex-data}. This problem is exascerbated by the fact that important information is often embedded in rich content (images, audio, video) present in these ads. While a majority of web content consists of either plain or rich text \cite{mphillips-EOT2012}, the need for computer vision and pixel-based analysis has increased with the rise of multimedia. Recent work on Apache Tika is continuing to expand capabilities for detecting and analyzing textual information while also expanding its support into deep learning and computer vision frameworks. 

\subsection{Extending Tika into Computer Vision}

On the textual side, Tika has recently added support for information extraction tasks such as named-entity recognition (NER) using popular Natural Language Processing toolkits like Stanford CoreNLP\cite{Finkel:2005:INI:1219840.1219885}, Apache OpenNLP\cite{ApacheOpenNLP}, and MIT Lincoln Lab's MITIE \cite{MITIE-github}. On the computer vision side, and the focus of this paper, Tika now provides support for one of the most popular and well-documented deep learning frameworks, Google's Tensforflow. Specifically, our integration of Tensorflow is focused around object recognition, which allows for the identification of objects of interest in graphical data. 

Historically, the best object recognition systems were inaccruate, but this has changed due to recent advancements in deep neural networks, larger training datasets, and improved computing resources. Tensorflow is a scalable, Python-based system and it natively supports image recognition via its \em {Inception} model \cite{abadi2016tensorflow}. \em{Inception} provides a neural network trained on the ImageNet corpus \cite{krizhevsky2012imagenet}, a dataset of 14,197,122 images and classified using the WordNet taxonomy. The end result is a highly-scable off-the-shelf system that can accuractely identify and classify objects in images into a thousand categories. 

Integration of Tensorflow with Tika presented a significant challenge: Tensorflow does not provide out of the box bindings to Java based frameworks. Apache Tika is primarily written in Java and thus integrating with Tensorflow is not straight forward like any other JVM compatible libraries. In the following sections we explore the pros and cons of various methods of integration.

% Object recognition is a standard problem in computer vision which deals with the recognition of objects of interest in the graphical data. In the context of images it is often called as image recognition. Historically image recognition was a challenging task and its accuracy of the recognized objects were much lower than average Human performance. However, due to the recent advancements in deep neural networks and availability of larger datasets with faster computing resources, we now have systems which have nearer or better performance than average human beings\cite{karpathy-cnn-compare}.

% Today's deep learning frameworks are focused towards performance gain from native code and GPU optimization for fast matrix manipulations. One of the most popular and well-documented deep-learning systems is Google's Tensorflow \cite{abadi2016tensorflow}. Tensorflow is a scalable, Python-based system and it natively supports image recognition via its {\em Inception} model \cite{abadi2016tensorflow}. Inception provides a neural network trained on the ImageNet corpus \cite{krizhevsky2012imagenet}, a dataset of 14,197,122 images and classified using the WordNet taxonomy. As such, Tensor 

\subsection{Tools for Law Enforcement} \label{sec:memex-tools}
The ultimate goal of integrating and characterizing diverse content scattered across the web is to provide law enforcement analysts with tools that will help them quickly identify potentially illegal activity, and multimedia content is key piece of the puzzle. Images and videos often provide salient information not available in text; in many cases, illegal weapons dealers intentionally embed revealing details in rich content mediums because they are harder to identify. This thinking coincides with the rise of weapons trafficking on social media platforms such as Snapchat, YouTube, and Instagram, where communication is centered around images and video \cite{socialmedia}. The integration of Tensorflow and Tika provides a single, streamlined platform that unites the extraction of textual and rich content. This combined content can then be exposed through search and visualization interfaces that improve analysts' abilities to drill-down and explore comprehensive, diverse content contained in weapons ads (see Figure \ref{fig:interface-diagram}). 
%\begin{figure*}
%        \includegraphics[width=\textwidth,height=5cm]{tensorflow-tika-integration}
%        \caption{Tika and Tensorflow Integration}
%        \label{fig:tf-tika-integration}
%\end{figure*}

\section{INTEGRATION} \label{sec:integration}
To integrate Tika and Tensorflow, we extended Tika's \texttt{Recogniser} interface which was introduced as part of our work in integrating named entity recognition (NER) toolkits ddescribed in the prior section. Our new interface was called \texttt{ObjectRecogniser}. The goal of \texttt{ObjectRecogniser} is to facilitate multiple implementations that extend beyond Tensorflow and that may include other future deep learning and other recognition frameworks. The main component of the interface contract is a function that accepts image data and returns a list of \texttt{RecognisedObject}s.

For our initial implementation of \texttt{ObjectRecogniser} we created a python based command line (CLI) tool as an entry point to Tensorflow's image recognition network. This tool inspected the environment for its requirements which included the Tensorflow command line tool, and when its requirements were not met it failed and reported the failure. Apache Tika executed in the JVM process where -- as on every invocation of CLI tool -- a new native process was created and destroyed. Tika passed image path as a command line argument to the tool as shown in Figure \ref{fig:tf-tika-integration} (a). The tool parsed the arguments, then passed the content to the Tensorflow network and reported the results by printing it to standard output. Tika parser then read the result from its output stream. We did not extend Tensorflow's existing ImageNet/Inception model training and simply used it out of the box already configured in the Tensorflow python program.

The Java Native Interface (JNI) is vendors' recommended way of integrating native code libraries to Java frameworks\cite{gordon1998essential}. JNI acts as glue between bytecode instructions that run within the Java Virtual Machine (JVM) and the native code instructions that run directly on the CPU. Theoretically, this is the best way of merging the JVM world with the native code. At runtime, the bytecode of Tika (caller) and native code of tensorflow (callee) runs within a single process from the operating system's perspective as shown in Figure \ref{fig:tf-tika-integration} (b).

The developers of Tensorflow framework recommended using gRPC based integration for the production systems\cite{goog-tfserve}. gRPC is a client-server based architecture in which caller acts as an RPC client and callee serves as a server in a different address space. Unlike traditional RPC frameworks, gRPC is a high-performance, high-CPU, and bandwidth efficient transport on top of HTTP/2 that supports full duplex streaming\cite{about-grpc}. In our case, we embedded gRPC client in Tika JVM and exported Tensorflow image recognition capabilities as remote procedures via gRPC service. We used Tensorflow Serving, a gRPC server implemented in C++, and also created a docker container to host it.  Collecting the needed libraries to build the gRPC interface proved to be a non-trivial effort, and rather than require users of our Tika and Tensorflow integration to install these libraries, we also investigated building a Representation State Transfer (REST) interface \cite{Fielding:2000:ASD:932295}.

REST is a client-server architecture paradigm for connecting heterogeneous systems without the need of states \cite[Chapter~5]{Fielding:2000:ASD:932295}. The REST application programming interfaces (API) is powered by the HyperText Transfer Protocol (HTTP) which abstracts the complexities of Transmission Control Protocol.
We created REST API for Tensorflow image recognition using python Flask. The Flask based HTTP service registered a TCP port and offered HTTP API endpoints as shown in Figure \ref{fig:tf-tika-integration} (d), and the REST interface had the advantage of minimizing client dependencies for using our framework. REST clients are light-weight and have easily installable dependencies across all major programming languages. Our REST API endpoint accepted HTTP POST requests with image data in the request body. This service loaded the Inception v3 \cite{SzegedyVISW15} model during the initialization phase and held the model in memory for reusing it during the future HTTP Requests.

% \begin{figure}[h]
%     \centering
%     \includegraphics[width=\columnwidth]{military_dog}
%     \caption{Military Person with a German Shepherd Dog \newline Courtesy: Wikimedia Commons}
%     \label{fig:military-dog}
% \end{figure}

% The REST API, upon recognizing the objects in the image \ref{fig:military-dog}, returned the response in the JSON format with the following structure:

% \begin{lstlisting}[language=json, label=code:json-output,
% 	frame=single, xleftmargin=5.0pt, xrightmargin=5.0pt,
%     caption=JSON Response from REST API]
% {
%   "confidence": [ 0.362026, 0.130613],
%   "classnames": [
%     "German shepherd, alsatian",
%     "military uniform"
%   ],
%   "classids": [211, 866 ],
%   "time": {
%     "read": 1,
%     "units": "ms",
%     "classification": 257
%   }
% }
% \end{lstlisting}

On the other side of the system, we implemented the class \texttt{TensorflowRESTRecogniser} which used HTTP Client to communicate with the REST API. This implementation of \texttt{ObjectRecogniser} converted the image data to HTTP POST request with multipart form data and sent it to REST API. It parsed the JSON response to retrieve the object names, IDs and confidence scores. We also created a Docker specification for bootstrapping the Tensorflow image recognition REST API for semi-automated deployment of the system. We created the Docker to provide an easily usable client and server for Tensorflow Tika integration providing all needed dependencies and capabilities for the integration automatically.

%\includegraphics[scale=0.40]{tika-tflow-rest-design}

\section {EVALUATION} \label{sec:evaluation}

\subsection{Analysis of results}
\begin{figure}[h]
	\includegraphics[width=\columnwidth]{top10classes}
	\caption{Top 10 image classes found in the dataset}
	\label{fig:top10ImgClass}
\end{figure}
The image classifier model used in our experiments was Inception-V3 \cite{SzegedyVISW15}. Inception-V3 model was trained on ImageNet 2012 dataset which contains 1000 classes\cite{ILSVRC15}.
The 10 most frequently occurred classes and their frequencies are shown in the Figure \ref{fig:top10ImgClass}. Since the crawlers were focused to retrieve the web pages and linked images that are related to weapons classifieds, the top classes in our dataset were found to be `revolver', and `riffle'.

\subsection{Manual cross validation of results}
We cross validated the predictions by using a subset of human labeled images. The results are shown in Figure \ref{fig:uk-hack-eval}.  
%//TODO: Describe more if space permits
\begin{figure}[h]
	\includegraphics[width=\columnwidth]{uk-hack-evaluation2}
	\caption{Manual cross validation of results for the two weapon types : `revolver' and `rifle'}
	\label{fig:uk-hack-eval}
\end{figure}

\subsection{Integration techniques}
The pros and cons of all the integration techniques described in Section \ref{sec:integration} are shown in Table \ref{tab:int-technique}.
\begin{table*}[bt]
	\centering
	\begin{tabularx}{\textwidth}{
			|p{\dimexpr.10\linewidth-2\tabcolsep-1.3333\arrayrulewidth}% column 1
			|p{\dimexpr.45\linewidth-2\tabcolsep-1.3333\arrayrulewidth}% column 2
			|p{\dimexpr.45\linewidth-2\tabcolsep-1.3333\arrayrulewidth}|% column 3
		} \hline
		\textbf{Method} & \textbf{Pros} & \textbf{Cons} \\ \hline
		CLI 
		& \tabitem Easy to develop and test \newline
		\tabitem Easy to install dependencies \newline
		\tabitem Isolation of dependencies and concerns
		& 
		\tabitem Additional load on secondary storage to share image data. \newline
		\tabitem An extra process is created and destroyed for each call. \newline
		\tabitem Additional IO for each call since the model file is loaded and unloaded for each process  
		\\ \hline
		JNI
		& 
		\tabitem The most efficient utilization of resources. \newline 
		\tabitem No external IO required since the system is monolithic.
		& 
		\tabitem The JNI native glue code has to be produced for all the platforms. \newline
		\tabitem Utilities like \texttt{protobuf} should also be glued with JNI, because they are required for deserializing models\cite{javacpp-240}.
		\\ \hline
		gRPC
		&
		\tabitem An efficient way for integrating heterogeneous systems. \newline 
		\tabitem The client system is easily portable, server system can also be easily portable by making use of containerization or virtualization. 
		& 
		\tabitem The gRPC client depended on specific versions of transitive dependencies such as HTTP Client which conflicts with existing functionality based on older HTTP Clients.
		\\ \hline
		REST API
		&
		\tabitem An efficient way for integrating heterogeneous systems. \newline 
		\tabitem The technology and practices are popular among developers. \newline 
		\tabitem The underlying HTTP is stable and well documented.
		&
		\tabitem The performance is slightly lower than gRPC. \newline
		\tabitem Slightly higher bandwidth usage due to additional metadata introduced by HTTP to the packets.
		\\ \hline
	\end{tabularx}
\caption{Brief comparison of integration techniques}
\label{tab:int-technique}
\end{table*}

%%%%%%%% COMMENT BEGIN
\iffalse
\subsubsection{Command Line Invocation (CLI)} \label{sec:eval-cli}
The average time taken to recognize image in this method was 6 seconds.

The pros of Command Line Interface:
\begin{itemize}
	\item Easy to develop and test
	\item Easy to install the requirements
	\item Isolation of dependencies / concerns
\end{itemize}

The cons of this technique:
\begin{itemize}
	\item The image data is shared via secondary storage which puts additional load on the secondary storage.
	\item An independent process is created and destroyed for every parse call.
	\item The ImageNet model is loaded and unloaded for every parse call due to ephemeral nature of processes. The InceptionV3 model used in our experiments is approximately 200MB in size, thus 200MB of additional IO for each parse call.
\end{itemize}

%% 2. JNI
\subsubsection{Java Native Interface (JNI)} \label{sec:eval-jni}

The pros of this method are
\begin{itemize}
	\item The most efficient utilization of resources
	\item No external input-output required as the entire task runs in a single process.
\end{itemize}

The cons of this method are:
\begin{itemize}
	\item The JNI glue code has to be compiled and packaged for all the platforms.
	\item Utilities like \texttt{protobuf} should also be glued with JNI, because they are required for deserializing models\cite{javacpp-240}.
\end{itemize}

%% 3. RPC
\subsubsection{gRPC Remote procedure Call (gRPC)} \label{sec:eval-rpc}

The pros of this method are
\begin{itemize}
	\item An efficient way for integrating heterogeneous systems.
	\item The client system is easily portable, server system can also be easily portable by making use of containerization or virtualization.
\end{itemize}

The cons of this method are:
\begin{itemize}
	\item The gRPC client depended on specific versions of transitive dependencies such as HTTP Client which conflicts with existing functionality based on older HTTP Clients.
\end{itemize}

%% 4. REST
\subsubsection{Representational State Transfer (REST) Application Programming Interface (API)} \label{sec:eval-rest}

The average REST API call was 260 milliseconds when the client and server were in the same host (i.e., the connection via loopback interface).

The pros of this method are
\begin{itemize}
	\item An efficient and popular way for integrating heterogeneous systems.
	\item The technology and practices are popular among developer community.
	\item The underlying HTTP is stable and well documented.
\end{itemize}

The cons of this method are
\begin{itemize}
	\item The performance is slightly lower than gRPC.
	\item Slightly higher bandwidth usage due to additional metadata introduced by HTTP to the packets.
\end{itemize}
\fi  %% COMMENT END

\section{CONCLUSION} \label{sec:future}
% The recent increase of content analysis technologies and architectures makes the systemization of such tools difficult.
Our motivations for the integration of image forensics into content analysis stems from a large corpus of 1.4 million weapons related images from the DARPA Memex effort, and our goals of automatically performing image classification to identify the illegal sale of automatic weapons and other dangerous objects on the web. We integrated the widely-used Google Tensorflow toolkit, and its ImageNet/Inception v3 model with the Apache Tika framework for automated and efficient image classification and analysis. We qualitatively and quantitiatvely evaluated the feasibility of our integration and report on running the integration over the Memex weapons data. We also describe our process for integrating Tensorflow and Tika. 



\section*{ACKNOWLEDGMENT}
This effort was supported in part by JPL, managed by the California Institute of Technology on behalf of NASA, and additionally in part by the DARPA Memex/XDATA/D3M programs and NSF award numbers ICER-1639753, PLR-1348450 and PLR-144562 funded a portion of the work. 

\bibliographystyle{ACM-Reference-Format}
\bibliography{MFSEC17.bib} 
\balance
\end{document}
