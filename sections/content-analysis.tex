% \section{CONTENT ANALYSIS IN MEMEX PROGRAM} \label{sec:memex}
% Resource-constrained law enforcement agencies are further challenged by the fact that important information is often embedded in rich content (images, audio, video) present in weapons ads. Generally, while a majority of web content consists of either plain or rich text \cite{mphillips-EOT2012}, the need for computer vision and pixel-based analysis has increased with the rise of multimedia. In additional to textual analysis approaches, Tika now provides support for one of the most popular and well-documented deep learning frameworks, Google's Tensforflow. As menioned in Section 1, our initial integration of Tensorflow is focused around object recognition, which allows for the identification of objects of interest in graphical data. 

% Object recognition is a standard problem in computer vision which deals with the recognition of objects of interest in the graphical data. In the context of images it is often called as image recognition. Historically image recognition was a challenging task and its accuracy of the recognized objects were much lower than average Human performance. However, due to the recent advancements in deep neural networks and availability of larger datasets with faster computing resources, we now have systems which have nearer or better performance than average human beings\cite{karpathy-cnn-compare}.

% Today's deep learning frameworks are focused towards performance gain from native code and GPU optimization for fast matrix manipulations. One of the most popular and well-documented deep-learning systems is Google's Tensorflow \cite{abadi2016tensorflow}. Tensorflow is a scalable, Python-based system and it natively supports image recognition via its {\em Inception} model \cite{abadi2016tensorflow}. Inception provides a neural network trained on the ImageNet corpus \cite{krizhevsky2012imagenet}, a dataset of 14,197,122 images and classified using the WordNet taxonomy. As such, Tensor 

% \label{sec:memex-tools}
% The ultimate goal of integrating and characterizing diverse content scattered across the web is to provide law enforcement analysts with tools that will help them quickly identify potentially illegal activity. Images and videos often provide salient information not available in text; in many cases, illegal weapons dealers intentionally embed revealing details in rich content mediums because they are harder to identify. This thinking coincides with the rise of weapons trafficking on social media platforms such as Snapchat, YouTube, and Instagram, where communication is centered around images and video \cite{socialmedia}. The integration of Tensorflow and Tika provides a single, streamlined platform that unites the extraction of textual and rich content. This combined content can then be exposed through search and visualization interfaces that improve analysts' abilities to drill-down and explore comprehensive, diverse content contained in weapons ads (see Figure \ref{fig:interface-diagram}). 